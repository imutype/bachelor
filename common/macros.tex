%%%%%%%%%%%%%%%%%%%%%%%%%%%%%%%%%%%%%%%%%%%%%%%%%%%%%%%%%%%%%%%%%%%%5
%%% Packages

\usepackage{geometry}
\usepackage[utf8]{inputenc}
\usepackage[T1]{fontenc}
\usepackage[pdftex]{graphicx}
\usepackage[ngerman]{babel}
\usepackage{colortbl}
\usepackage{soul}
\usepackage{textcomp}
\usepackage[usenames,dvipsnames]{xcolor}
\usepackage[toc,page]{appendix}
\usepackage{amssymb}
\usepackage{amsmath}
\usepackage{enumerate}
\usepackage{tikz}
\usepackage[hidelinks]{hyperref}
\usepackage{lastpage}
\usepackage{nameref}
\usepackage{etoolbox}
\usepackage[binary-units=true]{siunitx}
\usepackage{eurosym}
\usepackage{wrapfig}
\usepackage{multirow,bigdelim}
\usepackage{multicol}
\usepackage{pgfplots}
\usepackage{listings}
\usepackage{array}
\usepackage[within=none]{caption}
\usepackage{adjustbox}
\usepackage{xspace}
\usepackage[inline]{enumitem}
\usepackage{booktabs}
\usepackage{tabularx}
\usepackage{subcaption}
% \usepackage{float}
\usepackage{titlesec}
\usepackage[backend=biber,labelnumber,defernumbers=true,style=authoryear,natbib=true]{biblatex}
\usepackage{makecell}
\usepackage{minted}
\usepackage{nicefrac}
\usepackage{csquotes}
\usepackage{lipsum}
\usepackage{rotating}
\usepackage{pdflscape}
\usepackage{afterpage}
\usepackage{ltablex}
\usepackage{keycommand}

\let\tmp\oddsidemargin
\let\oddsidemargin\evensidemargin
\let\evensidemargin\tmp
\reversemarginpar

% rubber: setlist arguments --shell-escape
%%%%%%%%%%%%%%%%%%%%%%%%%%%%%%%%%%%%%%%%%%%%%%%%%%%%%%%%%%%%%%%%%%%%5
%%% Helper macros for different types of TODOs

\newcommand{\factcheck}{\textcolor{orange}{\textsuperscript{\textbf{!!}}}~}
\newcommand{\citationneeded}{\textcolor{blue}{\textsuperscript{\textbf{??}}}~}
\newcommand{\cn}{\citationneeded} % alias for \citationneeded
\newcommand{\todo}{\textcolor{green}{TODO!}}

%%%%%%%%%%%%%%%%%%%%%%%%%%%%%%%%%%%%%%%%%%%%%%%%%%%%%%%%%%%%%%%%%%%%5
%%% References

\newcommand{\figlabel}[1]{\label{fig:#1}}
\newcommand{\chaplabel}[1]{\label{chap:#1}}
\newcommand{\seclabel}[1]{\label{sec:#1}}
\newcommand{\tablabel}[1]{\label{tab:#1}}
\newcommand{\subseclabel}[1]{\label{subsec:#1}}
\newcommand{\lstlabel}[1]{\label{lst:#1}}

\newcommand{\figref}[1]{\figurename~\ref{fig:#1}}
\newcommand{\chapref}[1]{\chaptername~\ref{chap:#1}}
\newcommand{\secref}[1]{\secname~\ref{sec:#1}}
\newcommand{\tabref}[1]{\tablename~\ref{tab:#1}}
\newcommand{\subsecref}[1]{\subsecname~\ref{subsec:#1}}
\newcommand{\lstref}[1]{\listingscaption~\ref{lst:#1}}

\newcommand{\Figref}[1]{\figref{#1} ,,\nameref{fig:#1}''}
\newcommand{\Chapref}[1]{\chapref{#1} ,,\nameref{chap:#1}''}
\newcommand{\Secref}[1]{\secref{#1} ,,\nameref{sec:#1}''}
\newcommand{\Tabref}[1]{\tabref{#1} ,,\nameref{tab:#1}''}
\newcommand{\Subsecref}[1]{\subsecref{#1} ,,\nameref{subsec:#1}''}
\newcommand{\Lstref}[1]{\lstref{#1} ,,\nameref{lst:#1}''}

% use for referencing a chapter inside a \cite, e.g. \cite[\citechap{5}]{mycite}
\newcommand{\citechap}[1]{Kapitel~#1}

% Optinally rename headings :)
%\addto\captionsngerman{\renewcommand{\chaptername}{Kap.}}
%\addto\captionsngerman{\renewcommand{\figurename}{Abb.}}
%\addto\captionsngerman{\renewcommand{\tablename}{Tab.}}
%\addto\captionsngerman{\renewcommand{\sectionname}{Abschn.}}
%\addto\captionsngerman{\renewcommand{\subsectionname}{U.-abschn.}}
\renewcommand{\listingscaption}{Codebeispiel}
\newcommand{\secname}{Abschnitt}
\newcommand{\subsecname}{Unterabschnitt}


%%%%%%%%%%%%%%%%%%%%%%%%%%%%%%%%%%%%%%%%%%%%%%%%%%%%%%%%%%%%%%%%%%%%5
%%% Tikz style rules and configuration

\usetikzlibrary{calc}
\usetikzlibrary{fit}
\usetikzlibrary{shapes}
\usetikzlibrary{arrows}
\usetikzlibrary{shapes.misc}
\usetikzlibrary{positioning}
\usetikzlibrary{shapes.geometric}
\usetikzlibrary{decorations.markings}
\usetikzlibrary{decorations.pathreplacing}
\usetikzlibrary{shadows}
\usetikzlibrary{automata}
\usetikzlibrary{matrix}
\usetikzlibrary{intersections}
\tikzset{
    keyboardkey/.style={
      draw,
      fill=white,
      drop shadow={shadow xshift=0.25ex,shadow yshift=-0.25ex,fill=black,opacity=0.75},
      rectangle,
      rounded corners=1pt,
      inner sep=2pt,
      minimum height=12pt,
      minimum width=1em,
      line width=0.5pt,
      font=\scriptsize\sffamily
    },
    onslide/.code args={<#1>#2}{% http://tex.stackexchange.com/a/6155/16595
        \only<#1>{\pgfkeysalso{#2}}
    },
    flowchart node/.style={
        rectangle,
        rounded corners,
        minimum width=3cm,
        minimum height=0.8cm,
        text width=3cm,
        align=center,
        draw=black,
        fill=white,
        thick,
        font=\scriptsize,
    },
    flowchart arrow/.style={
        thick,
        ->,
        >=stealth,
    },
}
\pgfplotsset{
    /pgfplots/xlabel near ticks/.style={
        /pgfplots/every axis x label/.style={
            at={(ticklabel cs:0.5)},
            anchor=near ticklabel
        }
    },
    /pgfplots/ylabel near ticks/.style={
        /pgfplots/every axis y label/.style={
            at={(ticklabel cs:0.5)},
            rotate=90,
            anchor=near ticklabel
        }
    }
}

%%%%%%%%%%%%%%%%%%%%%%%%%%%%%%%%%%%%%%%%%%%%%%%%%%%%%%%%%%%%%%%%%%%%5
%%% Text formatting commands

\newcommand{\imagesource}[1]{\par\vspace{2mm}\tiny{}Bildquelle: \url{#1}}
\newcommand{\productname}[1]{\textsc{#1}}
\newcommand{\fremdwort}[1]{\emph{#1}}
\DeclareRobustCommand\keyboard[1]{%
    \tikz[baseline={($(key) + (0, -3pt)$)}]\node[keyboardkey] (key) {\textbf{#1}};
}
\newcommand{\shift}{$\Uparrow$}
\newcommand{\return}{\rotatebox[origin=c]{180}{$\Rsh$}}
\newcommand{\spacebar}{\textvisiblespace{}}

\newcommand{\iic}{I\textsuperscript{2}C\xspace}

%%%%%%%%%%%%%%%%%%%%%%%%%%%%%%%%%%%%%%%%%%%%%%%%%%%%%%%%%%%%%%%%%%%%5
%%% Title formats

\titleformat{\chapter}{\normalfont\huge\bfseries}{\thechapter}{2em}{}
\titleformat{\section}{\normalfont\Large\bfseries}{\thesection}{1em}{}
\titleformat{\subsection}{\normalfont\large\bfseries}{\thesubsection}{1em}{}
\titleformat{\subsubsection}{\normalfont\bfseries}{\thesubsubsection}{1em}{}

%%%%%%%%%%%%%%%%%%%%%%%%%%%%%%%%%%%%%%%%%%%%%%%%%%%%%%%%%%%%%%%%%%%%5
%%% Spacing

\setlength{\parindent}{0pt}
\setlength{\parskip}{1.5ex}
\newcommand{\only}[1]{}

%%%%%%%%%%%%%%%%%%%%%%%%%%%%%%%%%%%%%%%%%%%%%%%%%%%%%%%%%%%%%%%%%%%%5
%%% Layouting rules

% Don't spread text vertically on the page
\raggedbottom
% Try not to split footnotes
\interfootnotelinepenalty=10000
% \floatplacement{figure}{tb}

%%%%%%%%%%%%%%%%%%%%%%%%%%%%%%%%%%%%%%%%%%%%%%%%%%%%%%%%%%%%%%%%%%%%5
%%% Itemization style

\renewcommand{\labelitemi}{--}
\renewcommand{\labelitemii}{$\rightarrow$}
\renewcommand{\labelitemiii}{$\bullet$}
\renewcommand{\labelitemiv}{$\bullet$}

% \renewcommand{\labelenumi}{\theenumi.}
% \renewcommand{\labelenumii}{(\theenumii)}
% \renewcommand{\labelenumiii}{\theenumiii.}
% \renewcommand{\labelenumiv}{\theenumiv}

%%%%%%%%%%%%%%%%%%%%%%%%%%%%%%%%%%%%%%%%%%%%%%%%%%%%%%%%%%%%%%%%%%%%5
%%% Font family

% \renewcommand{\familydefault}{\sfdefault}
\sisetup{math-rm=\mathsf,text-rm=\sffamily,per-mode=symbol}

%%%%%%%%%%%%%%%%%%%%%%%%%%%%%%%%%%%%%%%%%%%%%%%%%%%%%%%%%%%%%%%%%%%%5
%%% Color defintiions

\definecolor{uhhred}{RGB}{226,0,26}
\definecolor{primary}{RGB}{226,0,26}
\definecolor{secondary}{RGB}{78,78,78}

% Some plot colors, same scheme that matplotlib uses by default
\definecolor{plot0}{HTML}{0072bd}
\definecolor{plot1}{HTML}{d95319}
\definecolor{plot2}{HTML}{edb120}
\definecolor{plot3}{HTML}{7e2f8e}
\definecolor{plot4}{HTML}{77ac30}
\definecolor{plot5}{HTML}{4dbeee}
\definecolor{plot6}{HTML}{a2142f}


%%%%%%%%%%%%%%%%%%%%%%%%%%%%%%%%%%%%%%%%%%%%%%%%%%%%%%%%%%%%%%%%%%%%5
%%% Bibliography
\addbibresource{../common/sources.bib}

\DeclareFieldFormat{labelnumberwidth}{\mkbibbrackets{#1}}

% \usepackage{xpatch}
% \xpretobibmacro{author}{\mkbibbold\bgroup}{}{}
% \xapptobibmacro{author}{\egroup}{}{}
% \xpretobibmacro{bbx:editor}{\mkbibbold\bgroup}{}{}
% \xapptobibmacro{bbx:editor}{\egroup}{}{}
% \renewcommand*{\labelnamepunct}{\mkbibbold{\addcolon\space}}
\setlength{\bibitemsep}{1ex}
% \setlength{\bibhang}{3ex}

\newcommand{\letbibmacro}[2]{%
  \csletcs{abx@macro@#1}{abx@macro@#2}%
}
\letbibmacro{original-cite}{cite}

\renewbibmacro*{cite}{%
  \printtext[bibhyperref]{%
    \printfield{labelprefix}%
    \ifentrytype{online}
      {\printfield[labelnumberwidth]{labelnumber}}
      {\usebibmacro{original-cite}}}}

\defbibenvironment{bibliographyNUM}
  {\list
     {\printtext[labelnumberwidth]{%
        \printfield{prefixnumber}%
        \printfield{labelnumber}}}
     {\setlength{\labelwidth}{\labelnumberwidth}%
      \setlength{\leftmargin}{\labelwidth}%
      \setlength{\labelsep}{\biblabelsep}%
      \addtolength{\leftmargin}{\labelsep}%
      \setlength{\itemsep}{\bibitemsep}%
      \setlength{\parsep}{\bibparsep}}%
      \renewcommand*{\makelabel}[1]{\hss##1}}
  {\endlist}
  {\item}

\assignrefcontextkeyws[sorting=none]{online}

%%%%%%%%%%%%%%%%%%%%%%%%%%%%%%%%%%%%%%%%%%%%%%%%%%%%%%%%%%%%%%%%%%%%5
%%% Code listings
\usepackage[scaled]{beramono}
\setminted{
    framerule=1pt,
    xleftmargin=5mm,
    baselinestretch=1.0,
    style=colorful,
    linenos,
    fontsize=\small,
}
\let\oldinputminted\inputminted
\usepackage{mdframed}
\renewcommand{\inputminted}[2]{%
    \begin{mdframed}%
        \oldinputminted{#1}{#2}%
    \end{mdframed}%
}
