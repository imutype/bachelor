\chapter{Ziele} \chaplabel{ziele}

In diesem Kapitel zeige ich die notwendigen Eigenschaften des zu entwerfenden
Systems auf und begründe ihre Relevanz.

\section{Kompatibilität zur klassischen Tastatur}

Damit das Produkt eine taugliche Alternative zu einer klassischen Tastatur
darstellt, muss es mindestens die gleiche Anzahl Eingaben wie diese
unterscheiden können.  Ebenso soll es Hilfstasten für Tastenkombinationen
erkennen, sodass bestehende Software nicht angepasst werden muss. Wir möchten
ausdrücklich mehr erreichen als nur eine beschränkte Eingabe zu erkennen, etwa
limitiert auf die 26 Zeichen des lateinischen Alphabets.

Wir versuchen also, Tastenanschläge der Tastatur zu emulieren. Hierbei geht es
uns nicht um die Wirkung, die diese Taste softwareseitig auslöst. Zum Beispiel
wird das Tastaturlayout von unserem System nicht beachtet -- stattdessen
arbeiten wir wie eine normale USB-Tastatur mit Tastencodes, welche vom System
dann einer beliebigen Wirkung zugewiesen werden können. Dies bedeutet im
Besonderen, dass wir zwischen Groß- und Kleinbuchstaben nicht unterscheiden.
Diese Unterscheidung findet erst im Betriebssystem statt, durch Interpretation
der Shift-Taste.

\section{Verwendung von maschinellem Lernen}

Wie bereits in der Zielsetzung des Projektes festgehalten, soll das System
Algorithmen des maschinellen Lernens (ML) verwenden, um Rückschlüsse auf die
gemachten Tastatureingaben zu ziehen. Dies ist keine willkürliche Entscheidung.
Wir halten dieses Projekt für einen geeigneten Anwendungsfall für maschinelles
Lernen.

Ein Grund dafür ist, dass ein ML-Algorithmus viele Lerndaten benötigt, um darin
Muster zu erkennen und zu erlernen und danach selbstständig Vorhersagen treffen
zu können. Im Fall von überwachtem ML benötigt man zu den Eingabedaten auch die
tatsächlichen Zielwerte (\emph{ground truth}). Bei unserer Anwendung sind wir
in der Lage, eine große Menge Lerndaten mit Zielwerten relativ einfach
aufzuzeichnen, da wir die Handbewegung beim Tippen auf einer normalen Tastatur
zusammen mit den tatsächlichen Tastenanschlägen ermitteln können.

Außerdem sind ML-Algorithmen in der Lage, komplexe Zusammenhänge implizit zu
lernen und wiederzuerkennen. Dazu gehören auch solche Zusammenhänge, die ein
Programmierer bei der Entwicklung eines traditionellen Ansatzes nicht beachten
würde. Ein Beispiel dafür ist die folgende Eigenart, welche ich an mir selbst
beim Tippen beobachten konnte: Wenn ich den Buchstaben
\keyboard{B}\footnote{Wir verwenden für die gesamte Arbeit Tastaturen mit
US-Layout. Die Testperson bin ich selbst -- es sei angemerkt, dass ich zwar aus
dem Muskelgedächtnis tippe, dies jedoch kein sauberes 10-Finger-System ist.
Stattdessen ruhen die Finger meiner rechten Hand auf den Tasten HJKL, also eine
Taste nach links verrutscht. Diese Erklärung ist notwendig zum Verständnis der
später gezeigten Graphen und Experimente.} tippe, verwende ich dafür den
Zeigefinger der rechten Hand.  Zur gleichen Zeit bewegt sich mein linker
Zeigefinger ein Stück nach links oben, um Platz zu machen. Dies ist ein
Zusammenhang, der in meinen Bewegungsdaten erkennbar ist, und der für mich
persönlich zutrifft. Ein ML-Algorithmus dürfte in der Lage sein, diesen
Zusammenhang zu erkennen -- ein Programmierer hätte dies jedoch vermutlich
nicht als Regel im klassischen Algorithmus hinterlegt.

\section{Messung charakteristischer Werte}

Die Hardwarekomponenten des Systems müssen, damit das ML-Verfahren die
unterschiedlichen Bewegungen zuordnen kann, die charakterischen Werte für die
Bewegungen des Handapparates beim Tippen messen können. Im folgenden erläutere
ich die dafür relevanten Aspekte.

\subsection{Auswahl der Messwerte}

Zunächst soll die innere Konfiguration der Hand gemessen werden, also die
Position der Finger relativ zueinander und zu der gesamten Hand. Das liegt
daran, dass die Finger eines geübten Tastaturbenutzers die meiste Arbeit
erledigen. Je weniger langsame Handbewegungen nötig sind, desto schneller und
effizienter ist das Tippen.

Bei den Fingerbewegungen kann der Rollwinkel (Drehung um die Längsachse)
vernachlässigt werden, da die menschliche Hand zu dieser Bewegung nicht in der
Lage ist. Vertikale und horizontale Bewegungen sind jedoch sehr wohl möglich
und sollen erkannt werden, um benachbarte Tasten unterscheiden zu können.

Zudem müssen für einige Tastenanschläge die Hände bewegt werden, etwa um die
oberen Tastenreihen (Ziffern und Funktionstasten) oder weiter außen oder innen
liegende Tasten zu erreichen. Daher sollen auch Informationen über die
Handbasis verfügbar sein, etwa die Position, Beschleunigung oder Orientierung.

\subsection{Genauigkeit}

Die Tasten einer Tastatur sind relativ klein. Um unter zwei benachbarten Tasten
unterscheiden zu können, muss die Genaugkeit der gelieferten Daten hoch genug
sein. Aufgrund der vielen möglichen messbaren Größen können wir bezüglich der
Genaugkeit keine quantitativen Anforderungen stellen. Ich werden jedoch im
Analyseteil auf diese Fragestellung eingehen, und die ermittelten Daten darauf
untersuchen, ob benachbarte Tasten daran unterscheidbar sind.

\subsection{Datenrate}

Geübte Benutzer können etwa 200 bis 400 Anschläge pro Minute\footnote{vgl.
Statistik unter~\cite{typingspeed}} erreichen. Dies entspricht im Durchschnitt
einem Intervall \SI{150}{ms} bis \SI{300}{ms} pro Anschlag. Aus diesem Grund
muss die Datenrate ausreichend sein, um die komplette Bewegung zu erkennen.
Sowohl die Hardware- als auch die Softwarekomponenten müssen diese Datenrate
unterstützen. Wir haben uns eine Datenrate von \SI{100}{Hz} zum Ziel gesetzt,
erhalten also mindestens 15 Datenpunkte für jeden Tastenanschlag eines geübten
Benutzers.

\section{Geringe motorische Einschränkung}

Eine geringe motorische Einschränkung des Benutzers durch die Sensoren ist von
zweierlei Nutzen. Zunächst soll der Proband bei der Aufzeichnung der Lerndaten
sowie bei der tatsächlichen Nutzung die gleichen Bewegungen aus dem
Muskelgedächtnis abspielen. Dazu ist es notwendig, dass er in der Lage ist,
eben diese Bewegungen auch durchzuführen.

Auch bei der späteren Anwendung ist eine Einschränkung des Benutzers durch die
Sensoren nicht erwünscht. Ein alltagstaugliches Gerät sollte flexibel und leicht
sein, insbesondere wenn es die Befestigung von Hardware an der Hand erfordert.

\section{Haltungsunabhängigkeit}

Um bessere ergonomische Eigenschaften gewährleisten zu können, sollten die
Handbewegungen in jeder Körperhaltung durchgeführt werden können, zum Beispiel
im Stehen, Sitzen oder Liegen. Das System soll unabhängig von dieser
Körperhaltung die Eingaben erkennen können. Zum Tippen sollte auch keine
bestimmte Armhaltung vorgeschrieben sein. Müsste man beispielsweise die Arme
ausstrecken, um im Stehen zu tippen, würden diese schnell ermüden.  In diesem
Falle ist es eher wünschenswert, die Arme locker an den Seiten herunterhängen
zu lassen, und nur mit der Bewegung der Finger tippen zu können.

Zudem soll das System nicht ortsgebunden sondern mobil einsetzbar sein.
Vorstellbar wäre die Verwendung mit einem mobilen Endgerät (z.B. Smartphone)
oder künftig sogar ,,deviceless'', also nur in Kombination mit peripheren
Geräten wie einer AR-Brille.  Wünschenswert bei Verwendung von Hardware an
der Hand wäre die Möglichkeit, diese kabellos betreiben zu können.

\section{Prototyping-geeignete Software}

Der Entwurf des Systems beinhaltet nicht nur Hardwarekomponenten, sondern auch
die Architektur der verwendeten Software. Diese muss nicht nur die funktionalen
Anforderungen erfüllen, sondern soll in einer Weise strukturiert sein, die
unterschiedliche, transparente und wiederholbare Experimente erlaubt, und somit
gut für Prototyping geeignet ist.

\section{Ausbaufähigkeit zu marktfähigem Produkt}

Damit das entworfene System nicht nur ein Prototyp bleibt, sondern irgendwann
eine echte Alternative zur Tastatur für viele Benutzer werden kann, muss es von
vornherein darauf ausgelegt sein, zu einem marktfähigen Produkt entwickelbar zu
sein. Dafür sollten die Hardwarebausteine günstig sein, insbesondere in
Massenproduktion. Das fertige Produkt muss robust und alltagstauglich sein.

Es wäre wünschenswert, sowohl die Hardware als auch die Software soweit
generalisieren zu können, dass ein neuer Benutzer diese nicht aufwendig anpassen
muss. Wie in \secref{vereinfachungen} erwähnt lassen wir diesen Gesichtspunkt
in unserer Arbeit jedoch vorerst außer Acht.

% vim: tw=79
