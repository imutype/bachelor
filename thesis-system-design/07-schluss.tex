\chapter{Fazit} \chaplabel{fazit}

In diesem Kapitel fasse ich meine Arbeit zusammen und gebe einen Ausblick auf
weitere mögliche Forschungsthemen.

\section{Zusammenfassung}

In der vorliegenden Bachelorarbeit habe ich den Entwurf eines Systems zur
Aufzeichnung der charakteristischen Handbewegungen beim Tippen sowie der
dazugehörigen Tastatureingaben erläutert, und die Qualität dieses Entwurfes
bewertet. Ich bin außerdem auf die Nutzbarkeit eines solchen Systems in
Verbindung mit einem Verfahren des maschinellen Lernens als Alternative zur
klassischen Tastatur eingegangen.

Zunächst stellte ich die Kriterien auf, welche das System erfüllen müsse, und
erläuterte deren Relevanz. Besonderer Fokus lag hierbei auf der Auswahl der für
die Handbewegung beim Tippen charakterischen Werte, und der Art und Weise, wie
diese ermittelt werden könnten.

Im State of the Art stellte ich relevante Projekte vor, welche ähnliche und
verwandte Themengebiete bereits erforscht haben.

Den Kern der Arbeit bildete das Systemdesign. Ich stelle vor, warum wir uns für
die Verwendung von IMUs als Sensoren im Allgemeinen und für welche genauen
Hard\-ware\-kom\-po\-nen\-ten im Speziellen wir uns entschieden haben. Außerdem
behandelte ich die Verbindung der Komponenten, die Befestigung der Sensoren an
der Hand mithilfe eines Handschuhs, und den Aufbau des dazugehörigen
Softwaresystems.

Nach einer kleinen Zusammenfassung der Arbeit von \citet{caro}, welche das
maschinelle Lernen in unserem Projekt behandelt, bewertete ich unseren Ansatz
in Bezug auf das Systemdesign und die Umsetzung des Hardwareteils. Ich stellte
aufgetretene Probleme dar, sowie wie mögliche Lösungsansätze für diese.

Zuletzt möchte ich einen kleinen Ausblick geben, welche Möglichkeiten sich aus
dem vorgestellten System ergeben.

\section{Ausblick}

Wir sehen viele Möglichkeiten für eine Weiterentwicklung dieses Projektes.

Wir können uns vorstellen, dass ein solches System die Entwicklung weg vom
Layout einer traditionellen Tastatur ermöglicht. Das System könnte
\emph{online-learning} implementierten, und somit lernen, mit Veränderungen in
den Bewegungen beim Tippen umzugehen. Verwendet man dann dieses System über
längere Zeit, wird der Benutzer automatisch seine Bewegungen verändern, und
irgendwann wird sich ein für ihn optimiertes Muster eingespielt haben, welches
er verwenden kann, und welches das System versteht. Hier sehen wir großes
Potential in allen Arten der Kommunikation mit Computern, sei es Texteingabe
oder andere Steuerung.

Im weiteren sehen wir Möglichkeiten für den Einsatz unseres Systems für die
Steuerung von Robotern. Hierfür müsste ein entsprechendes kinematisches
Handmodell benutzt werden, um aus den Sensordaten möglichst akkurat die
Handkonfiguration abzuleiten. Auch mit ML könnte hier gearbeitet werden --
komplexe Bewegungen, die der Mensch im Laufe seines Lebens erlernt hat, wie
etwa das Greifen von Gegenständen, könnten einem Robotor ,,vorgemacht'' werden,
sodass dieser hieraus ebenfalls lernen kann.

Auch in der Videospielebranche ist eine Anwendung denkbar. Wie wir feststellen
konnten, ist dort die virtuelle Realität ein aktuelles Thema. Hier hat sich noch
keine Steuerungsmethode durchsetzen können, und ein Datenhandschuh könnte die
Brücke schlagen zwischen echter und digitaler Welt, um mühelos und intuitiv mit
virtuellen Gegenständen interagieren zu können.

Wir haben gezeigt, dass unsere Vision nicht unrealistisch ist, und mit
entsprechendem Aufwand umsetzbar scheint. Wir sind uns sicher, dass viele
Menschen von einem auf unseren Prinzipien basierenden Texteingabegerät
profitieren könnten, und damit besser, effizienter und gesünder mit Computern
zu interagieren.
