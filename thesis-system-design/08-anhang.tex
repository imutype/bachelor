\chapter{Anhang}

\section{Quellenverzeichnis}
\printbibliography[nottype=online,heading=none,title=none]

\section{Online-Quellen}
\newrefcontext[sorting=none]
\printbibliography[type=online,heading=none,title=none,env=bibliographyNUM,resetnumbers]

\section{Abbildungsverzeichnis}
\makeatletter
\@starttoc{lof}
\makeatother

\section{Tabellenverzeichnis}
\makeatletter
\@starttoc{lot}
\makeatother

\section{Datenträger-Verzeichnis}
Dieser Arbeit ist ein digitaler Datenträger beigelegt. Darauf finden Sie die
Arbeit in digitalem Format, sowie begleitende Materialen:

\begin{itemize}[noitemsep]
    \item diese Arbeit, sowie die Arbeit von Carolin Konietzny
    \item den Quelltext der Handschuh-Firmware
    \item den Quelltext der Softwarekomponenten
    \item diverse Tools und Skripte zur Datenanalyse
    \item das Exposee, welches unserem Projekt zugrunde liegt
    \item die Folien unseres Kolloquiumvortrags (in englischer Sprache)
    \item Schaltplan und Leiterbahnlayout des ,,Shields''
    \item Fotos und Filme des fertigen Handschuhs
\end{itemize}

Nicht enthalten sind aus Speicherplatzgründen die Aufzeichnungen der
Experimente. Diese können bei Interesse vom Projektverzeichnis auf Github
heruntergeladen werden:

\begin{center}
    \url{https://github.com/imutype/bachelor}
\end{center}

