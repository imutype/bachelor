\frontmatter
\newgeometry{centering,left=2cm,right=2cm,top=2cm,bottom=2cm}
\begin{titlepage}
\includegraphics[scale=0.3]{UHH-Logo_2010_Farbe_CMYK.pdf}
\vspace*{2cm}
\Large
\begin{center}{\color{uhhred}\textbf{\so{BACHELORTHESIS}}}
\vspace*{2.0cm}\\
{\LARGE \textbf{Prototyp für eine virtuelle Tastatur basierend auf IMUs und maschinellem Lernen - Systementwurf}}
\vspace*{2.0cm}\\
vorgelegt von
\vspace*{0.4cm}\\
Paul Bienkowski
\end{center}

\vspace*{3.9cm}

\noindent
MIN-Fakultät \vspace*{0.4cm} \\
Fachbereich Informatik \vspace*{0.4cm} \\
Arbeitsbereich Technische Aspekte Multimodaler Systeme \vspace*{0.4cm} \\
Studiengang: Informatik \vspace*{0.4cm} \\
Matrikelnummer: 6415133 \vspace*{0.8cm} \\
Erstgutachter: Dr.~Norman Hendrich \vspace*{0.4cm} \\
Zweitgutachter: Florens Wasserfall

\end{titlepage}

\restoregeometry{}

\cleardoublepage
\hspace{0pt}
\vfill
\begin{center}
    \begin{minipage}{0.7\textwidth}
        \begin{center}
            \textsc{Kurzbeschreibung}
        \end{center}
        \vspace{-2ex}
        \noindent\rule[0.5ex]{\linewidth}{0.5pt}
        Mit dem Projekt ,,Prototyp für eine virtuelle Tastatur basierend auf
        IMUs und maschinellem Lernen'' entwickeln wir ein System, welches eine
        Alternative zur klassischen Tastatur als Texteingabemethode für
        Computersysteme darstellt, indem es die Bewegungen der Hand beim Tippen
        erfasst und mithilfe maschinellen Lernens wiedererkennt. In dieser
        Arbeit befasse ich mich mit dem Design dieses Systems, ausgeschlossen
        hiervon ist das maschinelle Lernen. Weiterhin zeige ich, wie die
        gesamte Umsetzung der Hardware sowie die Architektur der Software diese
        Anwendung ermöglicht. Die Befestigung von IMUs (\fremdwort{inertial
        measurement unit}) an der Hand mittels eines ,,Datenhandschuhs'' erwies
        sich für diese Anwendung als sehr geeignet. Ich bewerte das
        Gesamtsystem und komme zu dem Schluss, dass unsere Vision umsetzbar
        erscheint und das vorgestellte Systemdesign einen geeigneten Weg zu
        diesem Ziel darstellt.
    \end{minipage}
    \\[2cm]
    \begin{minipage}{0.7\textwidth}
        \begin{center}
            \textsc{Abstract}
        \end{center}
        \vspace{-2ex}
        \noindent\rule[0.5ex]{\linewidth}{0.5pt}
        In our project ,,Prototype for a virtual keyboard based on IMUs and
        machine learning'' we develop an alternative text input system for
        computers, capable of replacing a traditional computer keyboard, which
        records movements made by the hand while typing and recognizes these
        using a machine learning algorithm. In this work I cover the design of
        this system, excluding the machine learning part. I show the actual
        realization of the complete hardware as well as the software
        architecture supporting the system and evaluate the complete system. I
        consider the attachment of IMUs (inertial measurement units) to the
        hand using a ,,data glove'' to be suitable for this task. I conclude
        that our vision appears feasible, and that the presented system design
        provides the necessary means to reach our goal.
    \end{minipage}
\end{center}
\vfill
\hspace{0pt}
\pagebreak


% \setcounter{tocdepth}{1}

\chapter{Inhaltsverzeichnis}
\makeatletter
{
\renewcommand{\baselinestretch}{0.75}\normalsize
\@starttoc{toc}
}
\makeatother

\cleardoublepage
\chapter{Abkürzungsverzeichnis}
\vspace{-1cm}
{
\renewcommand\arraystretch{1.1}
\begin{tabularx}{\textwidth}{@{}>{\bfseries}rX@{}}
    \toprule
    3D & 3-dimensional \\
    AR & \fremdwort{augmented reality}, erweiterte Realität \\
    CNN & \fremdwort{convolutional neural network}, Faltungsnetz\\
    DoF & \fremdwort{degrees of freedom}, Freiheitsgrade\\
    EEPROM & \fremdwort{electrically erasable programmable read-only memory} (Speicherbaustein) \\
    EMG & Elektromyografie (Messung elektrischer Muskelaktivität)\\
    GND & \fremdwort{ground}, Masse, Bezugspotenzial \\
    GPIO & \fremdwort{general-purpose input/output} (universell einsetzbarer Kontakt auf einer Leiterplatte)\\
    HAR & \fremdwort{human activity recognition} (Erkennung menschlicher Aktivitäten aus Sensor\-daten)\\
    \iic & \fremdwort{Inter-Integrated Circuit} (serieller Datenbus)\\
    IMU & \fremdwort{inertial measurement unit}, inertiale Messeinheit\\
    IPC & \fremdwort{inter-process communication}, Interprozesskommunikation\\
    MEMS & \fremdwort{microelectromechanical system}, mikroelektromechanisches System\\
    ML & \fremdwort{machine learning}, maschinelles Lernen \\
    ROS & \fremdwort{Robot Operating System}\\
    SCL & \fremdwort{signal clock}, Taktleitung bei \iic \\
    SDA & \fremdwort{signal data}, Datenleitung bei \iic \\
    SERCOM & \fremdwort{serial communication interface} (in Atmel SAM D Mikrokontrollern)\\
    PWM & \fremdwort{pulse-width modulation}, Pulsweitenmodulation\\
    SMD & \fremdwort{surface-mount device}, oberflächenmontiertes Bauelement \\
    SPI & \fremdwort{Serial Peripheral Interface} (serieller Datenbus) \\
    UART & \fremdwort{Universal Asynchronous Receiver Transmitter} (Baustein zur Realisierung serieller Schnittstellen) \\
    UDP & \fremdwort{User Datagram Protocol} (Transportprotokoll) \\
    VCC & \fremdwort{voltage at the common collector} (Versorgungsspannung) \\
    VR & \fremdwort{virtual reality}, virtuelle Realität\\
    YAML & \fremdwort{YAML Ain't Markup Language} (Datenformat) \\
    \bottomrule
\end{tabularx}
}
\setcounter{table}{0}

% Eigennamen:
% 24C64WP
% AK8975
% ATWINC1500
% BNO055
% GNU/Linux
% InerTouchHand
% InvenSense
% (NASA)
% RViz
% YAML
%
