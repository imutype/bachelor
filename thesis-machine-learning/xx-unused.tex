\subsubsection{ROS}

Wie wir bereits festgestellt haben, besteht das Projekt aus vielen einzelnen Komponenten. ROS (\emph{Robot Operating System}) ist auf Projekte dieser Art ausgelegt, da es im Kern ein System zur Kommunikation zwischen Prozessen ist. Mit den sogennanten \emph{Nodes} können die Komponenten unserer Software abgebildet werden. Die \emph{Topics} stellen die Kommunikationskanäle dar, die die Nodes verwenden, um Daten untereinander zu teilen. Dies erfolgt in Form von \emph{Messages}. Die Messages können ein eigen definierter Datentyp sein, welche ROS serialisieren und deserialisieren kann und welche man als C++- und auch als Pythonobjekte verwenden kann.Die Topics sind eine Form von Streams, dies ist besonders im Hinblick auf die Sensordaten von großem Nutzen, da wir hier mit Echtzeitdaten arbeiten.

Die Nodes melden sich über den sogenannten Master an und kriegen mitgeteilt, wer auf ihre Daten hören will und von wem sie die ihre benötigten Daten bekommen. Die Nodes können dynamisch dazu- oder abgeschaltet werden. Wenn also eine Node ausfällt, bricht nicht das ganze System zusammen. Dadurch erreichen wir eine Entkopplung der einzelnen Komponenten. Trotzdem sind die Schnittstellen der Komponenten durch die Topics klar definiert.

Da ROS dafür eingesetzt wird, das Schreiben von Robotersoftware zu vereinfachen\cite{web:ros} und unser Projekt viele Überschneidungen mit Aufgaben aus der Robotik besitzt, gibt es einige von unsbenötigte Funktionalitäten die in Form von Nodes bereits von ROS angeboten werden. Hierzu zählen unter anderem das Plotten von Daten und deren 3D-Visualisierung. Eine weitere von ROS bereitgestellte Funktionalität ist die Aufzeichnung und das spätere Abspielen der von den Topics übertragenenen Messages. Dies können wir gut gebrauchen, um die Einstellungen des Machine Learnings anzupassen und in komprimierter Zeit mit dem Lernalgorithmus zu experimentieren. Hierbei können wir den Komponenten die Zeitabstände geben, die sie benötigen, ohne die Zeitabstände in Echtzeit abwarten zu müssen.

